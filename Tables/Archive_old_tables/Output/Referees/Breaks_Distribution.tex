\begin{table}[h]%
\setlength\tabcolsep{2pt}
\caption{Distribution of Breaks Taken in Period 2}
\label{tab:BreaksDistribution}
\begin{center}%
{\small\renewcommand{\arraystretch}{1}%
\begin{tabular}{lcccccccccc}
\hline\hline\noalign{\smallskip}
 & \multicolumn{10}{c}{Number of Breaks Taken} \\
 \cmidrule{2-11}
Task & \hspace{10pt} 0 \hspace{10pt} & \hspace{10pt} 1 \hspace{10pt} & \hspace{10pt} 2 \hspace{10pt} & \hspace{10pt} 3 \hspace{10pt} & \hspace{10pt} 4 \hspace{10pt} & \hspace{10pt} 5 \hspace{10pt} & \hspace{10pt} 6 \hspace{10pt} & \hspace{10pt} 7 \hspace{10pt} & \hspace{10pt} 8 \hspace{10pt} & \hspace{10pt} 9 \hspace{10pt} \\
\midrule
Slider  & 36.9\% & 10.8\% &  9.1\% &  5.7\% &  4.5\% &  6.3\% &  2.3\% &  4.0\% & 19.9\% &  0.6\%\\Creative  & 16.3\% & 11.6\% & 12.8\% & 16.3\% &  8.7\% &  8.1\% &  3.5\% &  5.2\% & 17.4\% &  0.0\%\\\hline\hline\noalign{\medskip}
\end{tabular}}
\begin{minipage}{\textwidth}
\footnotesize {\it Note:} This table reports the distribution of the number of breaks by task. 
To create an opportunity cost of working, we offered agents a time-out button. Each time an agent clicked the time-out botton, the computer screen was locked for 20 seconds, and 5 Taler (0.05 Euros) were added to the agent's payoff. 
The number of breaks taken refers to the number of times the time-out button is used. 
The time-out button was disabled for the final 20 seconds of each period. The maximum number of breaks therefore is 9 for participants who started their first break within the first second of the experiment and 8 for everyone else. 
The sample includes all agents from the Control Group as well as agents from treatment groups where the principal decided to institute the tournament/gift. Agents with negative reward decisions are not part of this analysis. 
\end{minipage}
\end{center}
\end{table}
