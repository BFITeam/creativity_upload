\begin{table}[h]%
\setlength\tabcolsep{2pt}
\caption{Average Number of Breaks Taken by Treatment and Period}
\label{tab:BreaksMeans}
\begin{center}%
{\small\renewcommand{\arraystretch}{1}%
\begin{tabular}{llcc}
\hline\hline\noalign{\smallskip}
\bf Task & \bf Treatment & \bf Breaks in & \bf Breaks in \\
                 &                       & \bf Period 1  & \bf Period 2 \\
\hline
\noalign{\smallskip}
Slider&Control&3.43&3.02\\
Slider&Gift&3.17&2.18\\
Slider&Tournament&2.39&0.55\\
Creative&Control&3.39&3.77\\
Creative&Gift&3.95&3.80\\
Creative&Tournament&3.43&1.85\\
\hline\hline\noalign{\medskip}
\end{tabular}}
\begin{minipage}{\textwidth}
\footnotesize {\it Note:} This table reports the average number of breaks by task, treatment, and period. 
To create an opportunity cost of working, we offered agents a time-out button. Each time an agent clicked the time-out botton, the computer screen was locked for 20 seconds, and 5 Taler (0.05 Euros) were added to the agent's payoff. Breaks refer to uses of the time-out button. 
In the Gift and Tournament treatment groups, the principals could choose to implement an unconditional monetary gift for all or of a tournament incentive (rewarding the top two performers out of their four agents) between Periods 1 and 2. 
The sample includes all agents from the Control Group as well as agents from treatment groups where the principal decided to institute the tournament/gift. Agents with negative reward decisions are not part of this analysis. 
\end{minipage}
\end{center}
\end{table}
