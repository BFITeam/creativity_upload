\begin{table}[h]%
\setlength\tabcolsep{2pt}
\caption{Period 2 and Period 3 Effects of the Feedback Treatment}
\begin{center}%
{\small\renewcommand{\arraystretch}{1}%
\begin{tabular}{lcc}
\hline\hline\noalign{\smallskip}
 & \bf Slider Task & \bf Creative Task \\
\hline\noalign{\smallskip}
\noalign{\smallskip}
Feedback & 0.184* & 0.182* \\
Period 2 & (0.108) & (0.102) \\
\noalign{\smallskip}\hline
\noalign{\smallskip}
Feedback & 0.118 & 0.171 \\
Period 3 & (0.132) & (0.129) \\
\noalign{\smallskip}\hline
\noalign{\smallskip}
Positive Relative & 0.303 & 0.319** \\
Feedback Period 3 & (0.196) & (0.148) \\
\noalign{\smallskip}
Negative Relative & -0.028 & 0.035 \\
Feedback Period 3 & (0.135) & (0.144) \\
\noalign{\smallskip}\hline
\noalign{\smallskip}
 Controls  & YES & YES \\
 Baseline  & YES & YES \\
 Intercept & YES & YES \\
\hline\hline\noalign{\medskip}
\end{tabular}
\begin{minipage}{\textwidth}
\footnotesize {\it Note:}  This table reports the estimated OLS coefficient estimates for the \textit{Feedback} Treatment for Periods 2 and 3. 
Performance is measured as the number of correctly positioned sliders in the slider task, as the score achieved in the  creative task, and as the amount transferred in the discretionary transfer tasks. 
Period 3 effects are also presented separately for those that learned that they did or did not belong to the 50\% top performers. 
Feedback effects for Period 2 are estimated using the same specification that we used in Column III in Table \ref{tab:EQ_Pooled_Results}. 
For the analysis presented here, we added observations from the two Creative Task with Discretionary Transfers treatments. 
Post-treatment effects for Period 3 are estimated using the same specifications as presented in Table \ref{tab:Period3} (pooled and split up by positive or negative feedback). 
For the Period 3 analysis, regressions are done separately for each task. \\
The estimation includes all agents from the Control Group as well as agents from treatment groups where the principal decided to institute the tournament/gift. Agents with negative reward decisions are not part of this analysis. 
Additional control variables are age, age squared, sex, location, field of study as well as a set of time fixed effects (semester period, semester break, exam period). 
Heteroscedastic-robust standard errors are reported in parentheses. Significance levels are denoted as follows: * p < 0:1, ** p < 0:05, *** p < 0:01. 
\end{minipage}}
\end{center}
\label{tab:Feedback}
\end{table}