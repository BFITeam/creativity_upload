\begin{landscape}
\begin{table}[h]%
\setlength\tabcolsep{2pt}
\caption{Descriptive Statistics on Raw Output and Breaks Taken in Period 2}
\label{tab:BreaksBreakdown}
\begin{center}%
{\small\renewcommand{\arraystretch}{1}%
\begin{tabular}{lccccccc}
\hline\hline\noalign{\smallskip}
& \multicolumn{3}{c}{\bf Slider Task} & & \multicolumn{3}{c}{\bf Creative Task} \\ \cline{2-4} \cline{6-8}
\bf Treatment & \bf Output & \bf Time Worked & \bf Output per Second && \bf Output & \bf Time Worked & \bf Output per Second \\
& & \bf (out of 180s) & \bf of Worktime & & & \bf (out of 180s) & \bf of Worktime \\
\hline
Tournament &    32.21 &   168.93 &     0.19&&    25.50&   143.00&     0.18 \\ 
Gift &    24.38 &   136.33 &     0.18&&    15.66&   103.93&     0.15 \\ 
Control &    19.78 &   119.67 &     0.17&&    15.89&   104.64&     0.15 \\ 
 \\
 \multicolumn{3}{l}{\textit{Log Difference}} \\
\hspace{10pt} Tournament &     0.49 &     0.34 &     0.14&&     0.47&     0.31&     0.16 \\ 
\hspace{10pt} Gift &     0.21 &     0.13 &     0.08&&    -0.01&    -0.01&    -0.01 \\ 
\hline\hline\noalign{\medskip}
\end{tabular}}
\begin{minipage}{1.2\textwidth}
\footnotesize {\it Note:} This table reports raw, unstandardized, average output, time spent working, and output per time worked. 
This refers to the number of correctly positioned sliders in the slider task and to the creativity score in the creative task (please refer to section 3.1 for a description of the scoring procedure). 
Time worked is the total time (180 seconds) less the number of breaks times the length of breaks (20 seconds). 
Output per second of worktime is the ratio of those two quantities. 
To create an opportunity cost of working, we offered agents a time-out button. Each time an agent clicked the time-out botton, the computer screen was locked for 20 seconds, and 5 Taler (0.05 Euros) were added to the agent's payoff. Breaks refer to uses of the time-out button. 
Log Difference is the log of the treatment group statistic less the log of the control group statistic. Log differences provide a sense of relative effect sizes. 
Numbers may not add up due to rounding. 
For simplicity, this analysis ignores baseline differences in performance. 
In the Gift and Tournament treatment groups, the principals could choose to implement an unconditional monetary gift for all or of a tournament incentive (rewarding the top two performers out of their four agents) between Periods 1 and 2. 
The estimation includes all agents from the Control Group as well as agents from treatment groups where the principal decided to institute the tournament/gift. Agents with negative reward decisions are not part of this analysis. 
\end{minipage}
\end{center}
\end{table}
\end{landscape}