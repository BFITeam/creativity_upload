

discussionw ith steve

explanations here

frist: statistical aberration

second explanation: something about scoring metric. Agnets wanted to reciprocate and they tried to do so by coming up with amazing idea.. instead of that metric rewarded. So we went back and asked RAs to look for amazing ideas.... 

\subsection{Analyses of Post-treatment Effects in Period 3}\label{chap:ex_post}
% noch einf�gen
% Mago et al, momentum paper: We do not find evidence for psychological momentum, i.e., momentum which emerges when winning affects players� confidence
% In addition to strategic momentum, players in a best-of-three contest may also exhibit �psychological momentum.� Folk psychology suggests numerous (and often contradictory) ways in which relative position in the contest can
% affect performance. While there is no single definition of psychological momentum, it is often based on the maxim �success breeds success,� i.e., winning a period affects players� confidence and makes them more likely to win the next period (Vallerand et al., 1988; Baker et al., 1994a, 1994b; Kerick et al., 2000; Dorsey-Palmeteer and Smith, 2004).

% literatur zu pos. affect erg�nzen?
% aus Henessey and amabile 2010
% The bulk of this research indicates that positive affect facilitates not only intrinsic motivation (e.g., Isen \& Reeve 2005) but also
% flexible thinking and problem solving even on especially complex and difficult tasks (see Aspinwall 1998, Isen 2000).
% Yet the affect creativity association is complicated. Kaufmann (2003a) refutes the mainstream argument that positive mood reliably facilitates creativity.
% Some studies have shown that positive mood may facilitate productivity but not quality of ideas (e.g., Vosburg 1998). Other researchers
% have found that although positive affect manipulations may enhance mood and reduce state
% anxiety, they do not necessarily enhance divergent thinking (e.g., Clapham 2001). Conflicting evidence comes from nonexperimental
% settings, as well. George \& Zhou (2002) found that, under certain specific conditions within an organization, negative affect
% can lead to higher creativity than positive



We now briefly turn to performance in Period 3 in order to assess 
whether the gift or the tournament had any long-lasting effects on performance.
The analysis of Period 3 is particularly interesting for the 
\textit{Tournament} treatment as any crowding out of intrinsic motivation 
 might have been dominated by the tournament's incentive effect in Period 2 (see \citealp{bowles12JEL} for a recent overview). 
Also, Period 3 performance 
can reveal whether winning or losing a tournament has an independent  effect on subsequent 
performance, as subjects in the \textit{Tournament} 
 treatment learned at the end of Period 2 whether they did or did not belong to the best 50\% (winners/losers) 
in Period 2 and, hence, whether or not they won the tournament prize. Period 3 endowments 
were identical to those in Period 1 and subjects were aware that there would be no further rewards or incentives.

Analogous to the analysis of the main treatment effects in Period 2 (see Equation \ref{eq:reg}), we assess Period 3 performance by comparing the change in treatment group performance from Period 1, with that of the control group.\footnote{The results are robust to including Period 2 performance, using fixed-effects  linear panel models or random-effects models. } This allows us to account for effects associated with learning and exhaustion. Table \ref{tab:Expost} shows the results. 


% XXX Shana: Seiner Meinung nach sollten wir ansonsten unser ex post table so lassen wie es ist. Er findet die Pat sachen aber interessant und meinte wir sollten dazu eine Fussnote haben...also die Sachen im Maintext etwas herunterkochen auf "wie cool es gibt es post effekte und die sind getrieben von den Winnern".. das ist die main message und die ist ja mal interessant. dann in der Fussnote so was wie @: if one controls for output in Period 2, that coefficient is positive, and tournament coeff goes to zero, this suggests that part of the channel through which tournaments increase 3rd period output is by increasing 2nd period effort (momentum or learning); coefficient on winning th etournament also significant which suggests an independe: halo effect of winning" ... Steve would not call it joy of winning.. he thinks it could be anything: pressure, feeling observed, being in the spotlight etc. ... He would definitely use the second specification, i.e. the one that includes Period 2 for this. And then write “the pattern is similar in the creative task, albeit there is no statistically significant independent effect of winning for this task. Results available upon request”

For ease of interpretation, we separate the analyses by task. Columns I and II show the results for the slider task, and columns III and IV 
depict the results for the creative task. In each case, the first column (I and III) reports 
overall treatment effects. 
%\footnote{In this analysis we cluster standard-errors by individual as the error of a regression of performance in Period 2 (regressed on performance in period 1 and treatment dummies) is correlated with the error of a regression of  performance in Period 3 (using the same independent variables) by $\rho\approx 0.4$.} 
Interestingly, our main treatment effects carry over to Period 3 
even in the absence of further gifts or tournaments. 
The reciprocal effect of the gift in the slider task is statistically significant and even slightly larger 
in Period 3 than in Period 2. Hence, reciprocity in the slider task has a persistent effect. 
The same pattern holds in the \textit{Creative Transfer Gift} treatment 
(see columns III and IV of Table \ref{tab:Expost}). As was to be expected, there 
is no evidence for reciprocity in Period 3 in the original \textit{Gift} treatment (columns III and IV). 

  
In line with standard economic theory, performance in the \textit{Tournament} treatment is lower in Period 3 than in Period 2 due to the removal of the tournament incentive. It is noteworthy, however, that Period 3 performance exceeds Period 1 performance. This suggests that the \textit{Tournament} has a sustainable performance-enhancing effect. Columns II and IV reveal that the tournament winners drive this overall performance increase in Period 3. They work roughly 0.5 standard deviations harder than would-be winners in the control group. Interestingly, losing a tournament does not affect subsequent behavior. Losers' performance is either not statistically significantly different from their baseline performance or 
is even a bit higher. % the tournament winner effect is not sign different between slider and creative (wald test stata mit suest, p=0.57)


For real-world applications, it is important to understand the mechanism behind this 
positive \textit{winner effect}. 
One possible explanation is learning on the task. 
By definition, tournament winners have worked more and harder on the task than have other participants. 
To the extent that the tasks are subject to a steep learning curve, 
increased 
Period 2 effort and performance could translate into increased Period 3 performance 
even if Period 3 effort was the same for tournament winners and tournament losers. 
It is unlikely, however,  that this effect is driven solely by learning as performance in the control group increases only slightly between Periods 1 and 2 and is relatively similar in Periods 1 and 3 (see Figure \ref{fig:Bar_Chart_Round_1_3}). 
%We can, however, not rule out learning effects completely. 
Alternative explanations are that 1) the positive feedback associated with winning a tournament could heighten self-confidence and 
intrinsic motivation (\citealp{Eisenberger2003,
Vansteenkiste03}),
%\footnote{There is a substantial literature in economics documenting 
%the impact of performance feedback on performance with mixed evidence on the direction 
%of the effect. Examples include \cite{BlanesiVidal2011}, \citet{Azmat2010}, and \cite{Barankay2011b}. 
%These studies differ from what we look at here, as we isolate the effect of a one-time performance revelation 
%whereas  they look at feedback in repeated interactions, i.e. a combination of incentive and ex post effects.} 
or that 2) winning a tournament could put 
individuals in a positive mood, which would in turn positively affect their performance.\footnote{For a review of 
mood effects on performance see, for instance, \cite{Lane2005}. Related to mood effects, 
\cite{Kraekel2008} discuss the notion that tournaments may induce a ``joy of winning'' which in turn affects subsequent performance.
\cite{DeJarnette15} proposes a theory of effort momentum, with similar predictions.} 
These two  explanations differ with respect to whether the positive effect from winning 
should raise subsequent performance only for the task at hand (task-specific confidence or intrinsic motivation), or whether 
it should also spill over to a different, unrelated task (general mood effect). 

To shed light on this issue, we conducted two 
supplementary tournaments. In these supplementary tournament treatments, subjects were asked to 
work on both the routine and the creative task in alternating orders (either 
slider (Period 1) - creative (Period 2) - slider (Period 3) (\textit{SCS}) or 
creative (Period 1) - slider (Period 2) - creative (Period 3) (\textit{CSC}). 
Identical to the main \textit{Tournament} treatments,
agents received fixed wages in Periods 1 and 3, and the principal could
implement a tournament in Period 2. In total, 55 subjects participated in the
\textit{ Slider-Creative-Slider (SCS)} treatment, and 46 subjects
 in the \textit{ Creative-Slider-Creative (CSC)} treatment.
The right-hand  columns of Table \ref{tab:Summary_Statistics} show
summary statistics for these observations. There are no significant differences in
baseline performance between these two  supplementary tournament treatments
and the respective control groups. For this comparison, and for analyzing treatment effects below,
we use the creative task control group as a benchmark for
baseline performance and  Period 3 performance in \textit{CSC},
and the slider task control group to assess
performance in Periods 1 and 3 in \textit{SCS}. Note that we did not include
additional control groups with varying tasks but without the tournament.
We therefore cannot control for changes in Period 3 performance
that are caused by the change in the tasks per se, rendering our findings below suggestive
rather than conclusive. Table \ref{tab:reg_spillover} presents the results on Period 3 performance.
The results are split up by treatment and task.

Columns I and II report results from the slider task (overall and split up by winners and losers).
Columns III and IV do the same for the creative task. Analogously, Period 3 treatment effects on the
the mixed tasks \textit{SCS} and \textit{CSC} are depicted in columns V-VIII.
Each regression controls for baseline performance and compares effects to the respective control group.
We find no evidence for positive spillover effects -- neither overall nor for winners or losers separately.
The treatment effects in the mixed tasks treatments, \textit{SCS} and \textit{CSC} (Columns V-VIII) are positive
but small and statistically insignificant. Hence,  winning  a tournament or receiving positive feedback
may lead to higher subsequent performance, but we have suggestive evidence that such an increase is limited to the task at hand. Possible
mechanisms are increased task-specific self-confidence or intrinsic motivation. Our data does not lend
support for more general and task-unspecific effects on intrinsic motivation, such as mood effects.
% if one controls for output in Period 2, that coefficient is positive, and tournament coeff goes to %zero, this suggests that part of the channel through which tournaments increase 3rd period output %is by increasing 2nd period effort (momentum or learning); coefficient on winning th etournament %also significant which suggests an independent winning effect. the pattern is similar in the creative %task, albeit there is no statistically significant independent effect of winning for this task. Results %available upon request”

% Kindness associated with tournament:
% SNE: sollten wir unbedingt noch diskutieren .. wir hatten mal diskutiert, dass das unwahrscheinlich ist weil wir ja gezeigt haben, dass reciprocity bei der simplen task wichtig ist aber bei der kreativen nicht ... den pos. ex post effekt finden wir aber bei beiden tasks... ich bin nicht mehr so ueberzeugt von dieser Argumentation, weil wir ja nur gezeigt haben, dass unconditional gifts bei creative keine reziprozitaet hervorrufen ... aber ein conditional reward koennte das ja schon tun, oder??
% Shana, kann man die deutlich hoehere Erwartung des Prinzipal-Lohnes durch die Agenten als Gegenbeispiel anfuehren? Kann man die Intentionen der Arbeitgeber hier auffuehren, warum sie den Reward implementiert haben? Wir nehmen an, dass Reziprozitaet nur fuer die Routinetaetigkeit wirkt, daher sollte Reziprozitaet beim Turnier bei der Slidertask (und nur da) positiv auf den Anreizeffekt wirken waehrend Kreativ nur den Anreizeffekt auswirkt und somit sollte gelten Tournament (Slider) > Tournament (Kreativ), was aber nicht der Fall ist



\subsection{Looking into the Effectiveness of Tournaments}
\label{chap:tournament}

%Our results suggest that tournaments effectively elicit creative output from employees.
%For a variety of reasons it may, however, not always be desirable or feasible to institute tournaments. 
%Such reasons include sabotage among workers 
%(for instance, \citealp{Harbring2011}), self-selection of more risk-tolerant agents (for instance, \citealp{Eriksson2009,Dohmen2011a}, 
%or premature drop-out of low performers (for instance, \citealp{Berger2013}).
%For example, organizations may not always have 
%the financial means to provide a tournament prize.
%An alternative to a tournament with a financial prize is feedback. 
%Feedback typically comes free of cost, but can have a similar relative rank information structure 
%than tournaments. 
%An open question relates to the effectiveness of pure relative performance feedback vis a vis  
%a tournament with the same information structure but a financial prize. 
%To investigate this issue, we conducted a supplementary \textit{Feedback} treatment. This treatment also allows us 
%to study \ to what extent our findings in the 
%\textit{Tournament}  treatment were driven by a concern for a good relative standing rather than 
% by the incentive effect of the 
%financial reward itself.

Tournaments affect behavior via two different channels: 1) a concern for a good relative standing and 
2) the monetary incentive (the tournament prize). For policy, it is important to understand how much of the tournament effect 
is driven by the (costly) prize and how much is driven by (cheap) relative performance information. 
 To disentangle these two channels, we conducted a \textit{Feedback} treatment that conveyed
 the same information about relative rank than the \textit{Tournament} treatment 
but without monetary consequences.\footnote{There is a growing literature in economics that documents 
the impact of relative performance feedback on behavior. So far, the evidence is mixed 
with respect to the direction (positive or negative) of the effect. 
\citet{Azmat2010} and 
\citet{BlanesiVidal2011}, for example, find that feedback increases performance,
 whereas \cite{Barankay2011b} shows that performance rankings decrease performance. 
%In the psychological literature, \citep{Reeve1996,Deci99PB,Vansteenkiste03}, among others, argue 
%that positive feedback can enhance and negative feedback 
%can decrease intrinsic motivation and 
%feelings of competence, which in turn affects performance.
 } 


In the \textit{Feedback} treatment, the principal had to 
decide before the start of Period 2 whether or not  relative rank information would be provided 
to her agents at the end of Period 2. The provision of relative performance feedback was 
costless to the principal and payoffs in this treatment were identical to those in the 
control group in all three rounds. When the principals opted for feedback provision, 
agents were informed at the beginning of Period 2 that they would learn at the end of Period 2 
whether or not they belonged to the best 50\% of their group. 
Hence, the \textit{Feedback} treatment mirrored the information structure in the \textit{Tournament} treatment, 
but without monetary consequences. 
We observe 56 agents with a positive feedback decision in the slider task and 68 agents 
with a positive feedback decision in the creative task in this treatment.


Table \ref{tab:Feedback} displays the coefficient estimates of the feedback treatment from our main regressions. 
%(Equation (1) for Period 2 and Period 3 performance; Tables 
%\ref{tab:EQ_Pooled_Results} and \ref{tab:Expost}). 
We find that \textit{Feedback} increases performance by about 
0.18 standard deviations in both tasks. 
The coefficent estimates are marginally statistically significant in period 2.
The effect sizes are statistically significantly lower than those in the \textit{Tournament} 
treatment in both tasks (Wald-tests, p<0.00) and suggest that about one fourth of the 
increase in the \textit{Tournament} treatment is driven by
 a concern for a good relative standing while the remainder is driven by the 
desire to win the tournament prize. 


Effect sizes are of similar magnitude in period 3 but lose statistical significance. These overall effects hide, however, that 
individuals who received positive relative 
performance feedback perform significantly better than comparable others in the control group, 
mirroring our findings for tournament winners in the \textit{Tournament} treatment.\footnote{In fact, the coefficients on \textit{Tournament winner} and
\textit{Positive relative feedback} are statistically indistinguishable in both tasks (Wald-test creative task, 0.19; Wald-test simple task, p=0.25).} 
%This ex-post effect on individuals with positive feedback, 
%is however much smaller 
%than that of tournament winners (even relative to the original effect size) [CAN YOU CHECK THAT ARNE?].
%This finding is in line with  \citep{Eisenberger1999} who argues that 
%monetary rewards are a more salient recognition of good performance 
%than mere performance feedback, and thereby more effective in triggering
% feelings of competence and intrinsic motivation. 
Recipients of negative performance feedback, however, are not demotivated (in both the \textit{Feedback} as 
well as the \textit{Tournament} treatments). Their performance 
is  not statistically significantly different from their baseline performance. 


Taken together, the findings from both the \textit{Tournament} and \textit{Feedback} treatments 
suggest that the chance of receiving positive feedback enhances performance 
but that the effect is only strong when performing well is also rewarded with a monetary prize.
This holds for both tasks as well as for ex ante incentive effects and for overall post-treatment performance. 




%If the principal decided against the reward, agents and principal received 300 Taler as fixed endowment for Period 2 in the \textit{Gift} and \textit{Tournament} treatment, but principal and agent received only 100 and 600 in the control group

%While the majority of principals (88\% in the slider task and 94\% in the creative task) implemented the costless \textit{Feedback}, and abou
%feedback

