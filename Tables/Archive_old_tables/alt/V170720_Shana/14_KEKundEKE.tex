\section{Mixed Task Treatments}
\label{mixed}

To shed light into whether the positive post-treatment effect of winning the tournament 
is task-specific or stems from a general mood effect that should raise performance in any subsequent task, we conducted two 
supplementary tournament treatments. In these supplementary treatments, subjects were asked to 
work on both the routine and the creative task in alternating orders (either 
slider (Period 1) - creative (Period 2) - slider (Period 3) (\textit{SCS}) or 
creative (Period 1) - slider (Period 2) - creative (Period 3) (\textit{CSC}). 
Identical to the main \textit{Tournament} treatments, 
agents received fixed wages in Periods 1 and 3, and the principal could 
implement a tournament in Period 2. In total, 55 subjects participated in the 
\textit{ Slider-Creative-Slider (SCS)} treatment, and 46 subjects 
 in the \textit{ Creative-Slider-Creative (CSC)} treatment. 
The right-hand  columns of Table \ref{tab:Summary_Statistics} show 
summary statistics for these observations. There are no significant differences in 
baseline performance between these two  supplementary tournament treatments 
and the respective control groups. For this comparison, and for analyzing treatment effects below, 
we use the creative task control group as a benchmark for 
baseline performance and  Period 3 performance in \textit{CSC}, 
and the slider task control group to assess 
performance in Periods 1 and 3 in \textit{SCS}. Note that we did not include 
additional control groups with varying tasks but without the tournament. 
We therefore cannot control for changes in Period 3 performance 
that are caused by the change in the tasks per se, rendering our findings below suggestive 
rather than conclusive. 
Table \ref{tab:reg_spillover} presents the results on Period 3 performance. 
The results are split up by treatment and task.\\


%\begin{landscape}
\begin{table}[!htbp]
\caption{\label{tab:reg_spillover} Post-treatment  Effects of the Tournament in Period 3 by Task}%
\begin{center}%
{\small\renewcommand{\arraystretch}{0.7}%
\begin{tabular}{lcccccccc}
\hline\hline\noalign{\smallskip}
  & & & & & \multicolumn{4}{c}{\bf Mixed Tasks} \\
 	&   \multicolumn{2}{c}{\bf Slider Task} & \multicolumn{2}{c}{\bf Creative Task} &\multicolumn{2}{c}{\bf Slider-Creative-Slider} & \multicolumn{2}{c}{\bf Creative-Slider-Creative} \\
 & & & & & \multicolumn{2}{c}{\bf (SCS)} & \multicolumn{2}{c}{\bf (CSC)}\\	
 & I & II & III & IV & V & VI & VII & VIII \\
\hline\noalign{\smallskip}
Tournament & 0.245* &  & 0.340*** &  & 0.111 &  & 0.055 &  \\
 & (0.137) &  & (0.126) &  &  (0.109)&  & (0.121) &  \\[2mm]
Tournament Winner &  & 0.534*** &  & 0.449*** &  & 0.064 &  & 0.091 \\
 &  & (0.180) &  & (0.170) &  & (0.122) &  & (0.146) \\[2mm]
Tournament Loser &  & -0.030 &  & 0.236* &  & 0.158 &  & 0.007 \\
 &  & (0.146) &  & (0.134) &  & (0.142) &  & (0.132) \\[2mm]
Standardized Performance & 0.888*** & 0.848*** & 0.693*** & 0.670*** & 0.888*** & 0.891*** & 0.648*** & 0.648*** \\
in Period 1 & (0.060) & (0.063) & (0.080) & (0.086) & (0.052) & (0.053) & (0.089) & (0.090) \\ [2mm]
Intercept & 0.000 & 0.000 & 0.000 & 0.000 & 0.000 & 0.000 & 0.000 & 0.000 \\
 & (0.073) & (0.073) & (0.093) & (0.094) & (0.073) & (0.073) & (0.094) & (0.094) \\[2mm]
\hline\noalign{\smallskip}
Observations & 116 & 116 & 116 & 116 & 115 & 115 & 102 & 102 \\
 R-squared & 0.679 & 0.708 & 0.529 & 0.535 & 0.734 & 0.735 & 0.503 & 0.504 \\
\hline\hline
\end{tabular}}
\begin{minipage}{1.2\textwidth}
\footnotesize
\vspace{5mm}
{\it Note:} This table reports OLS estimates of standardized performances in Period 3. Performance is measured as the number of correctly positioned sliders in the simple task and as the score in the creative task, respectively.\\ 
The estimation includes all agents from the Control Group as well as agents from treatment groups where the principal decided to institute the tournament as well as agents from the supplementary treatments SCS and CSC. Agents with negative reward decisions are not part of this analysis. Heteroscedastic-robust standard errors are reported in parentheses. Significance levels are denoted as follows: * $p < 0.1$, ** $p < 0.05$, *** $p < 0.01$.
\end{minipage}
\end{center}
\end{table}
\end{landscape}

Columns I and II report results from the slider task (overall and split up by winners and losers). 
Columns III and IV do the same for the creative task. Analogously, Period 3 treatment effects on the 
the mixed tasks \textit{SCS} and \textit{CSC} are depicted in columns V-VIII. 
Each regression controls for baseline performance and compares effects to the respective control group.
We find no evidence for positive spillover effects -- neither overall nor for winners or losers separately. 
The treatment effects in the mixed tasks treatments, \textit{SCS} and \textit{CSC} (Columns V-VIII) are positive 
but small and statistically insignificant. Hence,  winning  a tournament or receiving positive feedback 
may lead to higher subsequent performance, but we have suggestive evidence that such an increase is limited to the task at hand. Possible 
mechanisms are increased task-specific self-confidence or intrinsic motivation. Our data does not lend
support for more general and task-unspecific effects on intrinsic motivation, such as mood effects. 