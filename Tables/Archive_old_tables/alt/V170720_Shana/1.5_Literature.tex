
\label{lit}

Simple, routine tasks can be defined as tasks that 
``can be accomplished 
by machines following explicit programmed rules.'' (\citealp{Autor03QJE}, p. 1283). 
By comparison, \cite{Amabile1997} defines creativity as  the 
production of ideas, solutions, and products that are novel (i.e., original) 
and appropriate (i.e., useful) in a given situation. This presumes that there is an alternative set of tasks that   allows subjects to engage in creative thinking. 

The economics literatures on incentives has almost exclusively focused on routine tasks 
confirming what standard economic 
theory predicts: financial incentives have a positive effect 
on performance in simple tasks because agents increase effort as long
as the benefits they derive from each additional unit of output exceed their 
effort costs. 
%\citep{Holmstrom91JLEO}. 
Positive incentive effects 
have  been demonstrated for different types of performance-dependent rewards 
such as piece rates, where workers are rewarded according to their absolute output
 \citep[for instance,][]{Lazear00AER} or tournaments, where workers are rewarded 
on the basis of their relative performance \citep{Harbring2003}.
%\footnote{In
 %the field, researchers observed work performance when, for instance,
%installing wind shields \citep{Lazear00AER}, 
%picking fruits \citep{Bandiera2005}, or planting trees \citep{Shearer2004};
%in the laboratory, subjects have been rewarded for typing letters \citep{Dickinson1999},
% cracking walnuts \citep{Fahr2000}, or filling envelopes \citep{Falk2006}.
%A main advantage of these tasks is that they offer a precise and easily 
%observable measure of the quantity (and the quality) of workers' output.} 
%But even as early as 1999, Prendergast \nocite{Prendergast99JEL} noted that these types 
%of  routine jobs are not very common. 

Similarly, the literature on gift exchange has also almost exclusively utilized simple, routine tasks. 
This literature documents that explicit financial incentives are not the only way 
to trigger workers' performance by showing that workers reciprocate wage gifts 
with higher effort \citep{Akerlof1982}.
The gift-exchange hypothesis has been tested and confirmed in a 
 myriad of laboratory experiments with both chosen and real effort 
(see, e.g., \citealp{Fehr1997} for an early study on the topic, or \citealp{Fehr00JEP} 
for an overview). There is, however, mixed evidence on the effectiveness of 
gift-exchange in the field (see, for example, \citealp{Gneezy06E}). 

Taken together, these findings inform human resource management on how 
to optimally reward employees in jobs that involve a clearly 
defined and repetitive workflow. Yet it is critical to understand 
whether these insights into the effectiveness of different rewards 
also hold for jobs that have a creative component.

Performance responses to incentives may differ, for example, because there is a difference 
in the degree of intrinsic motivation associated with working on these tasks,\footnote{
In psychology, the literature on how rewards affect creativity has centered around the 
question of how rewards affect intrinsic motivation -- motivation to engage in an 
activity out of interest, enjoyment, or a personal sense of challenge \citep{Amabile13}.
To date, the evidence in psychology is mixed with one camp of the literature documenting effects of rewards on intrinsic motivation and creativity (e.g., \citealp{Amabile1996, Joussemet1999, Deci99PB}) and with a second camp documenting positive effects (e.g., \citealp{Eisenberger2001, Eisenberger2003}).   \citet{Shalley2004}
and \citet{byron12} provide overviews of the literatures in psychology, 
education, and organizational studies. } because 
creative tasks tend to be more cognitively demanding, 
more risky, and of less
certain value than routine performances (e.g., \citealp{Amabile1996}; \citealp{eysenck95}). 
Many scholars as well as practitioners in this area have therefore suggested that motivating creative
performance is fundamentally different from motivating routine performance (e.g., \citealp{Amabile1996}). The nascent literature on economics on how rewards affect creative performance was mentioned above. Yet, to date, a direct comparison of responses to rewards across the two types of tasks is missing.  
 This paper addresses this gap.  



%There are two main opposing camps in this literature
% \citep{hennessey10}. On the one hand, the literature based on
%cognitive evaluation theory focuses on the controlling aspects of
%rewards and argues that rewards undermine
%intrinsic motivation and thus creative performance %\citealp{Deci85, amabile90, Amabile1996, Joussemet1999, Deci99PB}). 
 %On the other hand, Eisenbergers' general interest theory focuses on the informational aspects
%of rewards and argues that rewards provide behaviorally relevant information
%which increases performance especially
%in intrinsically motivated and creative tasks 
%\citealp{Eisenberger1997


%In economics, the possible undermining effects of rewards has been introduced 
%as crowding-out of intrinsic motivation (e.g., \citealp{Frey93EER, Frey94RS, Frey97AER}). 
  %This literature 
%has primarily focused on intrinsically motivated activities that are not creativity-oriented
%(e.g., \citealp{Frey01JES}, or \citealp{Gneezy2011} for overviews)	
%and still debates the existence of crowding out effects (e.g., \citealp{Fehr02EER, Charness09E, fang12}).





%A key difference between the two literatures is that they make
%opposite predictions for the effectiveness of rewards that are performance-contingent
%and those that are not. Deci and Amabile, their respective co-authors
%(e.g., \citealp{Deci85, amabile90}), and the related literature argue that
%performance-contingent rewards are perceived as controlling and limit the individual's sense of autonomy,
%which leads to a crowding-out of intrinsic motivation and thereby a reduction of
%performance.
%However, if rewards have a strong informational value, as for instance by providing relative performance feedback,
%the detrimental effects might be (partly) counterbalanced by effects of positive feedback. 
%Positive feedback is argued to enhance perceived competence and to increase intrinsic motivation.
%Therefore, more adequate control groups have been discussed in the literature, such as comparing reward groups with
%groups which receive the same information but without tangible rewards \citep{Deci99PB}.
%According to cognitive evaluation theory, the overall effect of tangible performance rewards finally depends on the relative
%strength of the negative controlling and positive information effect. However, tangible rewards are
% expected to lower intrinsic motivation compared to pure feedback without a material
%consequence \citep{Deci99PB}. Noncontingent rewards, on the other hand,
%should not affect intrinsic motivation as they render individuals' perception of autonomy untouched.

%Based on general interest theory, Eisenberger and his co-authors
% (\citealp{Eisenberger2003, Eisenberger96AP}),
%in contrast, suggest that noncontingent rewards that do not depend on individual
%performance crowd out intrinsic motivation as they are perceived
%to reward ``average'' or inadequate performance. Performance-contingent rewards, on the other hand,
%are perceived as honoring extraordinary achievements, which fosters (crowds in) intrinsic motivation.
% In particular, contingent rewards are claimed to
% strengthen feelings of autonomy and self-competence \citep{Eisenberger2003}.
%




% This 
%is especially likely for tasks which are cognitively complex or high in intrinsic motivation, 
%both of which are attributes of creative tasks \citep{Bonner2000,Camerer1999, Amabile1996, Shalley2004}.


%The addition of ``simple'' refines this latter definition by restricting 
%the set of tasks to those that are simple to understand and perform, 
%i.e., that do not require much instruction, skill, or prior knowledge.

