\section{Feedback Treatment}
\label{feedback}

%Tournaments affect behavior via two different channels: 1) a concern for a good relative standing and 
%2) the monetary incentive (the tournament prize). For policy, it is important to understand how much of the tournament effect 
%is driven by the (costly) prize and how much is driven by (cheap) relative performance information. 
% To disentangle these two channels, we conducted a \textit{Feedback} treatment that conveyed
% the same information about relative rank than the \textit{Tournament} treatment 
%but without monetary consequences. In the \textit{Feedback} treatment, the principal had to 
%decide before the start of period 2 whether or not  relative rank information would be provided 
%to her agents at the end of period 2. The provision of relative performance feedback was 
%costless to the principal and payoffs in this treatment were identical to those in the 
%control group in all three rounds. In case the principals opted for feedback provision, 
%agents were informed at the beginning of period 2 that they would learn at the end of period 2 
%whether or not they belonged to the best 50\% of their group. 
%We observe 56 agents with positive reward decisions in the Slider task and 68 agents 
%with positive reward decisions in the Creative task in this treatment.


