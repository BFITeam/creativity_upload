
There is a long tradition in economics that investigates 
the impact of explicit incentives (e.g., tournament rewards) and implicit rewards (i.e. wage gifts) 
on motivation and productivity 
in simple and routine tasks. Overall, these studies suggest that both types of incentives 
have a positive effect on performance.
To date, we have only very limited understanding of how performance in creative tasks responds to incentives.
As the share of workers performing tasks that require them to engage in non-routine problem solving 
and creative thinking has increased substantially 
over the last several decades \citep{Autor03QJE,Florida2002}, knowing how incentives affect creative tasks
has become increasingly important. 
 One 
particular challenge relates to incentivizing workers 
to perform well in these types of jobs. 
%\footnote{The nascent literature
 %on incentivizing creativity in economics will be discussed below. }
The purpose of this paper is to gain a deeper understanding of 1) how creative performance responds 
to rewards and of 2) whether 
the lessons learned from simple and routine tasks can be generalized to complex tasks 
that have a creative component.   


%Summary of study design and results
Towards this end, we conducted a laboratory experiment with more than 1000 subjects. In the experiment, 
subjects worked in groups of four agents for one principal who benefitted from the agents' efforts. 
Subjects
 worked on either a simple, routine task or on a multi-dimensional creative task that 
rewards the number as well as the originality of ideas, and were exposed to either a tournament 
incentive or a wage gift. In addition to those rewards, agents received a fixed wage.

A particular challenge is to capture creativity in a lab setting. 
We employ the Unusual Uses as a multi-dimensional, creative task.\footnote{The task is not new 
to the literature. In economics, it has 
 been used by \cite{dutcherJEBO12} in his study on the effects of telecommuniting on productivity. In creativity research, the 
task has been used by \cite{Shalley97}. }  In the task, 
subjects have to come up with as many and as original alternative uses for common objects such as a tin can or a 
sheet of paper. Hence, rather than measuring blue-sky creativity -- even though Unusual Uses 
does allow measuring the originality of ideas as well -- the task focuses on whether subjects can place 
common objects into a different context. This is a central element in business innovation and in 
corporate idea-suggestion systems\citep{woodman93}. In that sense, the paper complements the existing literature on creativity in economics that mostly 
focuses on either blue-sky creativity or on tasks that involve very little creativity 
such as those that involve pattern recognition. Another advantage of the task is that it measures and 
captures creativity along several different dimensions: quantity (the number of answers), 
breadth (the spread of answers across different idea categories), as well as originality
 (measured as either the statistical infrequency of answers or by subjective evaluation). 
The availability of these separate measures allows us to address issues such as quantity 
- quality tradeoffs when assessing the effect of incentives on creative performance.\footnote{Just like any other task, this task is, however, not without limitations. Please refer to Section \ref{XXX - experimental design} and Section \ref{conclusion} for a discussion.} 






%The experiment has three main treatments: a control treatment, a tournament incentive treatment, and a 
%wage gift treatment and two tasks: a mutidimensional creative task and a simple,  routine task.

%In the simple, routine task subjects work on the slider task. This is a real effort task 
%in which 
%subjects have to manually move sliders
%on a screen to a specified position with their mouse. The task is purely effort based. 
%We consider this task as a simple task, because it
%is  easy to understand and requires no prior knowledge. 





 
%The tasks were chosen to mimic the respective dominant scholarly definitions. 
%In the experiment, subjects 
%were randomly assigned to groups of five, with one principal (``employer'') and four agents (``employees'').
%We observe a baseline level of performance (period 1) without incentives as well as performance under 
%tournament incentives or following a wage gift (period 2).\footnote{We also observe subsequent performance in the absence 
%of another tournament or wage gift (period 3). In order to keep the paper short and concise, results on period 3 performance 
%were relegated to the appendix.}

%In all treatments, the principal's  payoff depended on the 
%output produced by their four agents, who received an exogenously assigned fixed wage in each period. 
%After agents worked on the task for one period, principals could decide whether or not to 
%provide additional rewards to their agents (at their own expense and without knowledge about agents' 
%performance in Period 1).
%In the \textit{Tournament} treatment, the principal could provide 
%an additional monetary prize to the 50\% best performing agents in her group. 
%In the \textit{Gift} treatment, the principal could opt for a
%monetary gift, half as large as the tournament prize, to all four agents in her group. 
%In the third period, subjects were asked to work for the principal for one more round
% without additional rewards.





%Hence, our design is 2 x 3: the routine and the creative task 
%with three
%treatments (control, gift, tournament) each. 
%We also ran a  host of supplementary treatments that serve to elucidate the underlying mechanisms. 



We find a strong positive performance response to the introduction of the tournament incentive
in both tasks
(routine and creative). The effect sizes are of similar magnitude in the two tasks, 
suggesting  that  performance in  both tasks is equally sensitive to competitive incentives.
% and 2) that there is little to no crowding out going on.  
%Subsequent to learning whether they were winners or losers at the end of Period 2,
%winners in the \textit{Tournament} outperformed comparable others in the control group in Period 3 even though
%there was no subsequent tournament prize at stake. Losers, on the other hand, returned to their 
%baseline level of performance. 
%This suggests 
%that performance-dependent rewards in the form of tournament incentives increase 
%creative performance and even have long-lasting effects on tournament winners.
Interestingly, however, the performance response to the wage gift
 differs between the two tasks. Subjects in the routine task
respond to the gift with an economically and statistically significant
increase in their performance. The effect size is similar to that typically found in the
literature on gift exchange (see, for instance, \citealp{Fehr02WP}).
However, there is no statistically significant effect of the gift on performance in the creative task.\footnote{While the coefficient is close to zero, it's the standard error is relatively large so we cannot exclude that the gift does have a small positive effect on performance. The effect size is, however, statistically significantly smaller than in the slider task. }


%This finding is surprising and warrants further 
%exploration. 
%There is prior evidence that reciprocity is reduced or even muted 
%when agents are uncertain about how their effort affects the principal's 
%payoff \citep{Hennig2010}. We wondered whether this could explain the 
%result as the value of creative performance is typically not completely clear 
%to either agents or  principals at the time that bonus decisions are made. 
%This is related to the fact that ideas take some time to be evaluated by, for example, 
%the market, and to be implemented in the form of innovations or new products. 
%In simple, routine tasks on the other hand, 
%total output and, hence, benefit to the principal are obvious to both parties. 

One possible explanation for the absence or reduction of reciprocity in the creative task could be 
the employees' uncertainty about how their efforts affected the 
principal's payoff in the creative task \citep{Hennig2010}.  
%\footnote{Leider and larkin show in adifferent context that implications of own behavior on 
%principal output have a large effect on strength of reciprocal response -- they show that 
%reciprocity is bigger when employer benefits more from eployees actin, ie. when effort is more efficient, which 
%does not speak directly to our case}
 Whereas agents had 
perfect control and knowledge over how many silders they positioned correctly, and, hence, 
how much profit they generated for the principal in the simple task, there was some uncertainty in 
the creative task due to the originality rating not being known to agents while they worked on the task.
% Hennigschmidt story is actually that in absence ofinfo on employerbenefit that subjects can tbe sure whether or not wage gift is fair!.. that is differnet from repayment story
%englmaier and leider (Gift Exchange in the Lab - It is not (only)
%how much you give) show that there is reciprocityonly when worker effort benefit manager a lot (no reciprocity when there is little). documenting that rather than a warm glow model
%where I want to work harder r becuase I got a gift.. workers are sensitive to information on and impact on principal payoff
%englaier and leider field experiment show that reciprocity only when employer receives a bonus if a lot of work gets done. they interpret this as suggesting that there is complementarity between
%wage gift and impact on employwer... another way to say would be that they have something tangible now to perceive as the profit function because the main task cannot really be felt
%What is more, complex and creative tasks in general 
%are characterized by this inability since the exact value of an idea to the principal is typically 
%somewhat uncertain and only becomes apparent with time, say, after an idea is implemented. 


%XXX READ THIS AGAIN AND POSSIBLY  INCLUDEXX Existing studies show that the reciprocal response to rewards depends on the size of the reward (XXX IS THIS TRUE).
%Hence, subjects seem to want to respond in proportion to the favor received. It is harder to 
%achieve this kind of fine-tuning in complex, creative tasks than in simple routine tasks, where output is 
%clearly observable.  This is less so in the tournament 
%treatment where subjects response should solely depend on their cost of effort relative to the tournament prize and the perceived likelihood 
%of winning it. Neither the beliefs about other subjects performance nor own effort costs should be affected by the 
%nature of the task.  Even though one could argue that 
%is important
%This is in line with evidence from previous studies showing that full information on payoffs 
%is important to a positive wage-effort relationship .

To test for this explanation, we decoupled subjects' efforts from the amount of profit that they generated for 
the principal. In particular, we informed subjects at the end of each round about 
the exact number of points that their suggestions had generated 
during the preceding round and then allowed them to decide on how many 
of those points they wanted to transfer to their 
principal. In this set-up we find a clear reciprocal response to the wage gift in the 
amount of points that subjects transfer to the principal. The effect size is very similar to the behavioral response in the 
simple task. This suggests that there is reciprocity in creative tasks similar to what is found in simple, routine tasks 
 when subjects 
have perfect control over how their actions affect the profit of the principal.

%Overall our results therefore suggest that creative tasks respond well to performance 
%incentives and that responses to rewards
%are very similar in creative and routine tasks; however, certain features inherent in 
%creative and other complex work affect the effectiveness of 
%gifts as rewards. Therefore, wage gifts for employees in creative tasks might
% not effectively boost creativity in practice. 

 
This study contributes to the small but growing literature in economics that studies the impact
 of rewards on creativity.\footnote{We use the term \textit{reward} for both the tournament reward scheme as well as the wage gift as a shorthand, even though the wage gift is not a reward as it is commonly understood, i.e. rewarding past performance.  Instead it is independent of both past and future performance. \cite{Fehr02EER}, for example, have therefore described wage gifts as implicit rewards. }  Previous studies have explored
%differs  in focus from the present study and looks, for instance,   at 
how creativity is influenced by the size of the reward \citep{Ariely2009b} or the type 
of  creative task \citep{charness12WP}. \citet{laske2015} focus on the multi-tasking aspect of creativity by
looking at how incentives affect quantity, quality, and novelty of creative output. 
  \citet{erat2015}, by comparison, 
compare the effectiveness of piece rate incentives and competitive incentives and find evidence for choking 
under pressure.\footnote{Another study in this realm is \cite{Eckartz2011}. 
They implemented a creative task as well as two control tasks for comparison 
(Raven's IQ and a number-adding task) in one experimental set-up. 
They find that neither the tournament incentive nor a piece rate  had any effect 
on performance in any of the three tasks. This makes it hard to draw clear conclusions 
about whether or not  rewards fail to enhance creativity, since their rewards did not affect 
their control groups either. A likely explanation is that baseline motivation was very high in all three tasks. } 


We extend this nascent economic literature on creativity as well as the literature
 on incentive provision in four distinct ways.
First, to our knowledge this is the first study to examine the effectiveness of 
financial gifts for increasing creative performance. The lack of literature on this subject is surprising given both the 
attention that gift exchange has received in the literature in the context of 
incomplete contracts (see \citealp{Fehr00JEP} for an overview) and the fact that
creative jobs seem to be a prime example of jobs that are complex, hard to monitor, and 
typically governed by incomplete contracts. 
%While some theories suggest that explicit financial 
%rewards might lower creative performance, 
%we know little about how intrinsic motivation interacts with implicit incentives such as wage gifts
% \citep{Fehr02EER}. 
Second, as far as we know, this is also the first study to compare 
the effect sizes of a performance-dependent, competitive incentive (tournament) with  that of a 
performance-independent wage gift in one set-up. By doing so, our design allows a direct comparison of 
the cost-effectiveness of the two reward schemes. Such a comparison is especially relevant for 
creativity as theory suggests that these two types of rewards 
might affect it in fundamentally different ways (e.g., \citealp{byron12}).
Third, we also advance the existing literature by studying the response to rewards in a 
simple and a creative task under the same experimental conditions. This allows us to speak directly to 
whether or not lessons learned from simple tasks can be generalized to creative or complex tasks without 
having to resort to comparisons across different studies (and, hence, different experimental conditions).
%Towards that end, we conducted a large-scale aboratory experiment which allows us to address these points. 
%(1) to compare the 
%effectiveness of a performance-dependent reward (tournament bonus) and a %performance-independent reward (wage gift), 
%and (2) to study their impact on performance in a creative 
%task as well as in a simple routine task. Thus, we are able to directly compare how %performance-dependent and
%performance-independent rewards affect creative and routine task performance in the %same experimental set-up. 
Fourth, we provide another data point to the discussion on whether or not it is possible to foster
 creative performance through financial incentives. 


The paper is structured as follows. Section \ref{lit} presents an overview of the existing literature. Section \ref{chap:experiment} 
describes the experimental set-up, the tasks, and the treatments. 
Section \ref{chap:results} presents our main results, Section \ref{chap:suppl} investigates mechanisms and looks into a number of 
supplementary issues, for instance, the absence of reciprocity in the creative task, the mechanism via which tournaments increase effort, and post-treatment effects.
Section \ref{chap:discussion} concludes.

%decide what we want to highlight as major findings and implications:
%This finding has far-reaching implications for economic theory
%that has so far mostly neglected to take factors such as the type of task and the degree of
%intrinsic motivation into account for predicting the effectiveness of reward schemes.
%Future research is needed to shed further light into these issues.XXX








