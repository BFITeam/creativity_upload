%Over the last decades many routine tasks have become computerized and 
%employees increasingly  face  complex tasks many of which involve creative thinking. 
%What is more, companies need to increasingly rely on an 
%active workforce that ``thinks along'' and provides 
%their employer with a stream of ideas and improvement suggestions 
%in order to gain or retain a competitive edge.  
%This raises the question of how to design reward schemes that foster creativity and the production 
%of novel and useful ideas. While there is a large body of research on incentives for 
%simple and routine tasks showing that explicit and implicit (gifts) financial rewards have positive 
%effects on performance, we know very little about how to motivate creativity. The 
%only exception is the literature on motivational crowding out that argues 
%that performance-dependent, financial incentives can have detrimental effects for tasks 
%that rely on intrinsic motivation. This is relevant here as creative tasks 
%are thought of as being intrinsically motivating. This literature has, however, never 
%looked at wage gifts.  


XXXhighlight that task rewards both quantity and originalizty dimension and that both contribute to the score in the task
XXX talk about how imortant originality is for understanding results.. is this paper really about routine versus creative or about routine versus any type of non-routine. 
XXX point that many and original ideas might not necessarily be "best" idea and in many settings principal may only be interested in the best idea... nevertheless, many original ideas might be a great input for development of best idea. from ref report... use almost word for word: Yet in most real-world settings, including examples like the ones cited in the paper (e.g., innovation inducement prizes, software coding contests, architecture competitions, etc.), what matters is product quality or performance, not originality, and quality is typically what gets rewarded: the agent with the “best” idea wins, not the agent with the “most” or “most original” ideas. Creativity may be a necessary input to developing a good idea, but it’s not an end in and of itself.

XXX insert something that says that it is surprsing that reciprocity goes away /is reduced so much!! one would expect reduction but not necessarily eleminiation of reciprocity. 


we learn how incentive affect creativity in a task with a known score metric, in real world people also often understand different dimensions on which they will be rewarded; if this is the best point and rewarding scheme is beyond the scope of this paper

-	There has been very little research in creativity, .. so anything about generalizability would be pure conjecture

how relevant is setting in which p cares about output of each individual agent rather than only about best idea (examples),

academics is good example where high number of original ideaas at outset of career because we don’t know for sure what idea might turn out to be valued by market; 
in graduate class that; important in ccases in which as you come up with ideas one does not know final value. 
Go after a lot and crazy ideas … .free disposal

 is it a problem that more ideas lead to more creative ideas?




This paper reports the results from a large-scale laboratory experiment that studies the impact of both explicit  incentives (tournaments) and implicit rewards (wage gifts)
on creativity. 
To the best of our knowledge, this is the first study to analyze the impact of wage gifts on creativity. We also have not come across another study that compares the effectiveness of wage gifts and tournaments in one set-up, providing insights into the relative effect sizes of these two rewards on both a creative and a simple task. The inclusion of the simple task allows us to benchmark the effectiveness of the different rewards on creativity and to link our results on creativity to the existing literature on tournament incentives and wage gifts that uses simple tasks. 

We report two sets of interesting results. The first relates to the tournament incentive.
Our results suggest that a tournament prize for above-average performance has a 
substantial positive incentive effect on creativity. The effect size is similar to that on performance in the simple task. This indicates that incentives can influence creativity and that there is no crowding out of intrinsic motivation.\footnote{\citet{erat2015}
 find evidence for choking under pressure in their creative task that is more ``blue skye'' in nature.} 
 About one fourth of this effect seems to be driven by a concern for relative rank, 
as is suggested by a supplementary treatment in which 
subjects work towards receiving the same rank information as in the tournament
but without financial consequences. Thus, it is largely  the monetary prize 
that is responsible for the positive  incentive effect of the tournament. Interestingly, the tournament 
does not only increase the quantity of ideas submitted, but also their quality in terms of 
originality. Further, tournament winners continue to outperform their own baseline 
performance and agents in the control group. A set of supplementary 
treatments suggests that this is driven by an increase in task-specific motivation that does not spill over to other tasks. 
Losers of the tournament show no signs of demotivation in both tasks. 
The effectiveness of tournaments is in line with the observation that creative tasks are often 
organized in a tournament framework in the real world.  For example, architects on virtually 
all major projects are chosen via a winner-take-all competition. 
The same is true on most innovation platforms such as innocentive.com that many 
companies now utilize for creative input. 

A second set of interesting results relates to the financial gift. We find an 
asymmetry in  its effectiveness between the two tasks. 
While the gift effectively triggers a reciprocal response in the 
simple, routine task, there is no evidence for reciprocity in the creative task. 
This suggests that the incentive response function differs between creative and simple tasks, 
despite the similarity of responses in the \textit{Tournament} treatment. 
Interestingly, 
this asymmetry holds for both positive and negative reciprocity, and principals 
seem to have anticipated this asymmetry as many more principals opted 
for the gift in the simple task than in the creative task. 
We explore a set of explanations for this finding and, through the implementation of additional treatments, can trace 
the effect back to agents' lack of knowledge about how exactly their effort 
translates into profit for the principal. While agents perfectly observe how much output 
they produce in the simple task, there is some uncertainty in the creative task because the value of their ideas to the principal depends, 
for example, 
on an originality rating. 
This is true for our creative task, but also holds for creative and complex tasks more generally. One implication of our finding therefore is that wage gifts might not boost performance in white-collar jobs that involve complex tasks and creativity. 
This is important to note as creative and complex tasks are typically governed
by incomplete contracts that might have rendered gift exchange a viable way of increasing effort.  
These results also speak to the ongoing debate about reciprocity in the lab versus in the field (e.g., \citealp{Kube2012a} 
or \citealp{Kessler13WP}) and
suggest that one reason for the observed absence of reciprocity in the field could be agents'
imperfect knowledge about how their effort affects their principal's profits. 

But even if potential benefits from ideas were fully transparent, 
our study suggests that wage gifts are less effective than 
tournaments in triggering agents' performance in both tasks. 
While the tournament was profitable for the principals that opted 
for it, the gift was not.\footnote{The payoff consequences of the \textit{Gift}
 in comparison to the control group (difference in average effort per work group minus costs of the gift) 
are -0.27 Euro in the slider task, -1.82 Euro in the creative \textit{Gift} treatment and 
-0.29 Euro in the treatment \textit{Creative Transfer Gift}. In the \textit{Tournament}, 
principals' payoff increased by 1.39 Euro in the slider and 1.45 Euro in the creative task.} 
% However, the assessment of reward's profitability depends on whether we compare it towards the principal's payoff in the control group (where no reward was available) or towards the principal's payoff in case he denies the reward.
In this study, tournaments clearly emerge as preferable to  wage gifts in terms of fostering creativity. Nevertheless, tournaments do have 
well-understood downsides that should be considered before implemention. For instance, tournaments have been shown to 
increase sabotage among workers (for instance, \citealp{Harbring2011}), to induce a 
self-selection of more risk-tolerant agents (for instance, \citealp{Eriksson2009,Dohmen2011a}), 
and to make low performers more likely to give up early in the contest (for instance, \citealp{Berger2013}).

Of course, our study has many limitations. Studying creativity is difficult. 
Limitations of the task : While there are Unusual Uses task has many virtues (quantifiable, well-established in creativity reserach, captures multiple dimensions of creativity), it is in the end a particular task
- unfamiliar task
- it is in a sense contrived (even though it is a proxy for creativity in teh real world) no one outside the lab actually works on this particular task 

Creativity is an amorphous concept and there is every reason to think that the particular context or challenge might affect

XXX Liitations of alternative uses task// 
from discussion with Steve: does not cover process in which creative solution to one problem; not put to market test (dollar value); 
this is obviously a ver specific task in a somewhat contrived setting relative to , say, in a place where blue skyp creativity is iportnat like in an advertising agency and only onen solution can be presented. But the virtue is that it is a task that is easy to communicate and understand with a straightforward incentive/payoff scheme that taps into the sorts of skills that might be important for understanding creativity outside the lab
fact that ratings are not tied to usefulness is a limitation (albeit we go halfway by not counting invalid uses)
does nto apply to tasks where quantity does not matter at all
does not mirror onein one a single creative task in real world but captures underlying processes of many
setting in which ideas can freely be disposed of. 

generalizability

section about how tournament m ight be more generalizeable becuse it can also be used when only best idea matters.
also talk about where directly applies: brainstorming sessions etc; what we wrote to ref 1

%For economic theory, the results also suggest that the type of task should be 
%incorporated as contextual factor in models studying the effectiveness of reward schemes.

The present study  calls for  future work that addresses whether the positive tournament 
effect that we find, also holds for contests with high-stakes. 
Previous studies indicate that high stakes cause choking under 
pressure and, hence, a non-linear relationship between reward size and effort, and that this is 
particularly true for cognitively challenging tasks \citep{Ariely2009b, Brach12MSa}.
Further, creative tasks come in many different forms. \citet{charness12WP} 
 show that responses to incentives might differ between tasks that have a clearly 
delineated solution and tasks that are open and require pure  out-of-the-box thinking. In our opinion, our task serves as a good proxy for everyday idea 
generation in firms. However, organizations also rely on other types of creative input that 
our task could not capture, such as  idea implementation or breakthrough 
innovations. 
Future work needs to test the robustness of our findings for  other kinds of 
complex and creative tasks. 
% XXX Ederer, Manso - Is Pay for Performance Detrimental to Innovation? Management Science, 2013: 
% Another issue relates to the problem that managers might refrain from pursuing innovative business strategies   
% as the latter offers not only the chance to increase profits but also bears the risk to suffer losses from these changes \citep{Ederer12MS}. 
% If managers are compensated according to pay-for-performance schemes that punish failures with low rewards, this may 
% in fact undermine innovation. Therefore, it can be beneficial to the firm to reward long term performance and forgive early failures 
% to motivate managers to engage in exploration.  
  

