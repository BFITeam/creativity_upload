\begin{landscape}
\begin{table}[h]%
\setlength\tabcolsep{2pt}
\caption{Descriptive Statistics on Raw Output and Breaks Taken in Period 2}
\label{tab:BreaksBreakdown}
\begin{center}%
{\small\renewcommand{\arraystretch}{1}%
\begin{tabular}{lccccccc}
\hline\noalign{\smallskip}
& \multicolumn{3}{c}{\bf Simple Task} & & \multicolumn{3}{c}{\bf Creative Task} \\ \cline{2-4} \cline{6-8}
\bf Treatment & \bf \hspace{5pt} Average \hspace{5pt} & \bf Time Worked &  \hspace{5pt} \bf Output per Second  \hspace{5pt} && \bf  \hspace{5pt} Average \hspace{5pt} & \bf Time Worked & \bf  \hspace{5pt} Output per Second \hspace{5pt} \\
& \bf Output & \bf (out of 180s) & \bf of Time Worked & & \bf Output & \bf (out of 180s) & \bf of Time Worked \\
\midrule
Performance Bonus &    32.21 &   168.93 &     0.19&&    25.50&   143.00&     0.18  \\ 
Gift &    24.38 &   136.33 &     0.18&&    15.66&   103.93&     0.15  \\ 
Control &    19.78 &   119.67 &     0.17&&    15.89&   104.64&     0.15 \vspace{5pt} \\ 
 \multicolumn{3}{l}{\textit{Log Difference}} \\
\hspace{10pt} Performance Bonus &     0.49 &     0.34 &     0.14&&     0.47&     0.31&     0.16  \\ 
\hspace{10pt} Gift &     0.21 &     0.13 &     0.08&&    -0.01&    -0.01&    -0.01  \\ 
\hline\noalign{\medskip}
\end{tabular}}
\begin{minipage}{1.2\textwidth} \setlength{\parindent}{15pt}
\footnotesize NOTE.--This table reports raw, unstandardized, average output, time spent working, and output per second of time worked. 
\textit{Average output} refers to the number of correctly positioned sliders in the simple task and to the creativity score in the creative task (please refer to Section 3.2.2 for a description of the scoring procedure in the creative task). 
\textit{Time worked} is the total time (180 seconds) less the number of breaks times the length of breaks (20 seconds). 
\textit{Output per second of time worked} is the ratio of those two quantities. 
To create an opportunity cost of working, we offered agents a time-out button. Each time an agent clicked the time-out button, the computer screen was locked for 20 seconds, and 5 Taler were added to the agent's payoff. \textit{Breaks} refer to the number of uses of the time-out button. 
\textit{Log difference} is the log of the treatment group statistic less the log of the \textit{Control} group statistic. Log differences provide a sense of relative effect sizes. 
Numbers may not add up due to rounding. 
For simplicity, this analysis ignores differences in Period 1 output. 
In the \textit{Gift} and \textit{Performance Bonus} treatment groups, the principals could choose to implement a performance-independent wage gift for all or a performance-dependent performance bonus (rewarding the top two performers out of their four agents) between Periods 1 and 2. 
Each estimation includes all agents from the \textit{Control} group and agents from treatment groups for which the principal decided to implement the performance bonus or gift. Agents from treatment groups for which the principal did not implement a reward are not included in this analysis. 
\end{minipage}
\end{center}
\end{table}
\end{landscape}