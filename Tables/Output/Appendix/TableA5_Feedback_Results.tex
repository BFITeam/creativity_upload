\begin{table}[h]%
\captionsetup{justification=centering}
\setlength\tabcolsep{2pt}
\caption{Treatment Effects on Period 2 and Period 3 Output for the Feedback Treatment}
\begin{center}%
{\small\renewcommand{\arraystretch}{1}%
\begin{tabular}{lcc}
\hline\noalign{\smallskip}
 & \bf Simple Task & \bf Creative Task \\
\hline\noalign{\smallskip}
\noalign{\smallskip}
Feedback & 0.184* & 0.182* \\
Period 2 & (0.108) & (0.102) \\
\noalign{\smallskip}\hline
\noalign{\smallskip}
Feedback & 0.118 & 0.171 \\
Period 3 & (0.132) & (0.129) \\
\noalign{\smallskip}\hline
\noalign{\smallskip}
Positive Relative & 0.303 & 0.319** \\
Feedback Period 3 & (0.196) & (0.148) \\
\noalign{\smallskip}
Negative Relative & -0.028 & 0.035 \\
Feedback Period 3 & (0.135) & (0.144) \\
\noalign{\smallskip}\hline
\noalign{\smallskip}
 Additional Controls  & YES & YES \\
 Period 1 Output  & YES & YES \\
 Constant & YES & YES \\
\hline\noalign{\medskip}
\end{tabular}
\begin{minipage}{\textwidth} \setlength{\parindent}{15pt}
\footnotesize NOTE.-- This table reports the estimated OLS coefficients for the \textit{Feedback} treatment in Periods 2 and 3. 
Output is measured as the number of correctly positioned sliders in the simple task, as the creativity score in the creative task (please refer to Section 3.2.2 for a description of the scoring procedure in the creative task), and as the amount transferred in the discretionary transfer treatments. 
Period 3 effects are also presented separately for those agents who learned that they were or were not among the top two performers. 
\textit{Feedback} effects for Period 2 are estimated using the same specification that we used in Table \ref{tab:EQ_Pooled_Results}, Column III. 
For the analysis presented here, we added observations from the two \textit{Creative Task with Discretionary Transfers} treatments. 
Post-treatment effects for Period 3 are estimated using the same specifications as presented in Table \ref{tab:Period3} (pooled and split up by positive or negative feedback). 
For the Period 3 analysis, regressions are done separately for each task. 

Each estimation includes all agents from the \textit{Control} group and agents from treatment groups for which the principal decided to implement the performance bonus, gift, or feedback. Agents from treatment groups for which the principal did not implement a reward are not included in this analysis. 
Additional control variables are age, age squared, sex, location, field of study, and a set of time fixed effects (semester period, semester break, exam period). 
Heteroscedastic-robust standard errors are reported in parentheses. 

*   $ p < 0.1  $

**  $ p < 0.05 $

*** $ p < 0.01 $
\end{minipage}}
\end{center}
\label{tab:Feedback}
\end{table}