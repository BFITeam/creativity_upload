\begin{table}[h]%
\setlength\tabcolsep{2pt}
\caption{Average Number of Breaks Taken by Treatment and Period}
\label{tab:BreaksMeans}
\begin{center}%
{\small\renewcommand{\arraystretch}{1}%
\begin{tabular}{llcc}
\hline\noalign{\smallskip}
\hspace{7pt} \bf Task \hspace{7pt} & \bf Treatment & \bf Breaks in & \bf Breaks in \\
                 &                       & \bf Period 1  & \bf Period 2 \\
\hline
\noalign{\smallskip}
Simple & Control & 3.43 & 3.02 \\
& Gift & 3.17 & 2.18 \\
& Performance Bonus & 2.39 & 0.55 \\
Creative & Control & 3.39 & 3.77 \\
& Gift & 3.95 & 3.80 \\
& Performance Bonus & 3.43 & 1.85 \\
\hline\noalign{\medskip}
\end{tabular}}
\begin{minipage}{\textwidth} \setlength{\parindent}{15pt}
\footnotesize NOTE.--This table reports the average number of breaks by task, treatment, and period. 
To create an opportunity cost of working, we offered agents a time-out button. Each time an agent clicked the time-out button, the computer screen was locked for 20 seconds, and 5 Taler were added to the agent's payoff. \textit{Breaks} refer to the number of uses of the time-out button. 
In the \textit{Gift} and \textit{Performance Bonus} treatment groups, the principals could choose to implement a performance-independent wage gift for all or a performance-dependent performance bonus (rewarding the top two performers out of their four agents) between Periods 1 and 2. 
The sample includes all agents from the \textit{Control} group and agents from treatment groups for which the principal decided to implement the performance bonus or gift. Agents from treatment groups for which the principal did not implement a reward are not included in this analysis. 
\end{minipage}
\end{center}
\end{table}